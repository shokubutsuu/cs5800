\documentclass[12pt]{article} 

\usepackage[top=1.5cm, bottom=1.5cm, left=1.5cm, right=1.5cm]{geometry}
\usepackage{color,graphicx}
\usepackage{amssymb}

\begin{document}  
\pagestyle{empty}
 




\vskip 0.5in


\begin{enumerate}

\item 
Suppose that $f(n) = \Theta(g(n))$. Assume that both functions increase without limit.

(a) Must it be true that $\log f(n) = \Theta(\log g(n))$? Prove or disprove.

(b)  Must it be  true that $2^{f(n)} = \Theta(2^{g(n)})$? Prove or disprove.\\


{\color{blue}{\bf Answer:\\}}

(a) 

Since $f(n) = \Theta(g(n))$, there exists constant $c_1$, $c_2$ and $n_0$ such that 

 $c_1\cdot g(n) \le f(n) \le c_2\cdot g(n),  \forall n \geq n_0$,

 then $log(c_1\cdot g(n)) \le log(f(n)) \le log(c_2\cdot g(n))$,

 $log(c_1) + log(g(n)) \le log(f(n)) \le log(c_2) + log(g(n))$,

 since $log(c_1)$ and $log(c_2)$ are constants,  
 
$\log f(n) = \Theta(\log g(n))$.\\


(b) 

as $f(n)$ and $g(n)$ goes larger, the value of $2^{f(n)}$ and $2^{g(n)}$ has significant different exponentials. 

As the exponential goes too big or too small, it goes against the defination of big Theta, where there's

$c_1\cdot g(n) \le f(n) \le c_2\cdot g(n),  \forall n \geq n_0$,

as the constant $c_1$ and $c_2$ will hardly make a difference in the exponential function.

Thus, this is not always true.


\newpage

\item
(a) Suppose you have a function of two variables, $n$ and $k$; for instance $h(n,k)$. \\ If you are told  that $h(n,k) = O(n+k)$, what should that mean mathematically? 

(b) 
 Let $f(n) = O(n)$ and $g(n) = O(n)$.  Let $c$ be a positive constant. \\   
Prove or disprove that  $f(n) + c\cdot g(k) = O(n+k)$. \\


{\color{blue}{\bf Answer:\\}}

(a)

Having $h(n,k) = O(n+k)$ means that function $h(n,k)$ grows depends on variables n and k, and there should exists constant c such that:

$c \cdot (n+k) \geq h(n,k) ,  \forall n \geq n_0, \forall k \geq k_0$.
\\

(b) 

$\because g(n) = O(n),$

$\therefore g(k) = O(k)$

$\because f(n) = O(n),$

$\therefore$ there exists constant $c_1$ and $c_2$, $n_0$, $k_0$ such that:

$c_1\cdot n \geq f(n), c_2\cdot k \geq g(k), \forall n \geq n_0, \forall k \geq k_0$

$\therefore c_1 \cdot n + c_2 \cdot k \geq f(n) + g(k)$

make $h(n,k) = f(n) + g(k)$,

$\therefore c_1 \cdot n + c_2 \cdot k \geq h(n,k)$.

when $c_1 \leq c_2$,

$ c_2 \cdot n + c_2 \cdot k \geq h(n,k)$.

$\therefore c_2 \cdot(n+k) \geq h(n,k)$, which fits the defination we defined previously.

same logic applies when $c_1 \geq c_2$.

$\therefore f(n) + c\cdot g(k) = O(n+k)$ is always true.






\newpage



\item  
Let $f(n) = \sum^n_{y{=}1} (n^6\cdot y^{23})$. 

Find a simple $g(n)$ such that $f(n) = \Theta(g(n))$, by proving that 
$f(n)=O(g(n))$, and that $f(n) = \Omega(g(n))$.

Don't use induction / substitution, or calculus, or any fancy formulas.  \\
Just exaggerate and simplify for big-O, then underestimate and simplify for $\Omega$.\\


{\color{blue}{\bf Answer:\\}}

$f(n) = \sum^n_{y{=}1} (n^6\cdot y^{23}) \leq n \cdot (n^6 \cdot n^{23})$
$f(n) \leq  n^{30}$

$\therefore$ there exists constants c, $n_0$ such that,

$ c \cdot n^{30} \geq f(n), \forall n \geq n_0$.

$\therefore f(n) = O(n^{30})$.

$f(n) = \sum^n_{y{=}1} (n^6\cdot y^{23}) \geq \sum^n_{y{=}{n/2}}n^6\cdot y^{23}$

$f(n)\geq \sum^n_{y=n/2}n^6 \cdot (\frac{n}{2})^{23}$

$f(n)\geq (\frac{n}{2})\cdot n^6 \cdot (\frac{n}{2})^{23}$

$=(\frac{1}{2})^{24}\cdot n^6 \cdot n^{24}$

$\therefore$ there exists constants c, $n_0$ such that,

$ c \cdot n^{30} \leq f(n), \forall n \geq n_0$.

$\therefore f(n) = \Omega(n^{30})$.

$\therefore f(n) = \Theta(n^{30})$.


\end{enumerate} 




\end{document}

