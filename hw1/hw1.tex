\documentclass[12pt]{article} 

\usepackage[top=1.5cm, bottom=1.5cm, left=1.5cm, right=1.5cm]{geometry}
\usepackage{color,graphicx}

\begin{document}  
\pagestyle{empty}
 




\vskip 0.5in


\begin{enumerate}

\item 
Suppose that $f(n) = \Theta(g(n))$. Assume that both functions increase without limit.

(a) Must it be true that $\log f(n) = \Theta(\log g(n))$? Prove or disprove.

(b)  Must it be  true that $2^{f(n)} = \Theta(2^{g(n)})$? Prove or disprove.\\


{\color{blue}{\bf Answer:\\}}

(a) 

Since $f(n) = \Theta(g(n))$, there exists constant $c_1$, $c_2$ and $n_0$ such that 

 $c_1\cdot g(n) \le f(n) \le c_2\cdot g(n),  \forall n \geq n_0$,

 then $log(c_1\cdot g(n)) \le log(f(n)) \le log(c_2\cdot g(n))$,

 $log(c_1) + log(g(n)) \le log(f(n)) \le log(c_2) + log(g(n))$,

 since $log(c_1)$ and $log(c_2)$ are constants,
 
$\log f(n) = \Theta(\log g(n))$.\\


(b) 





\newpage

\item
(a) Suppose you have a function of two variables, $n$ and $k$; for instance $h(n,k)$. \\ If you are told  that $h(n,k) = O(n+k)$, what should that mean mathematically? 

(b) 
 Let $f(n) = O(n)$ and $g(n) = O(n)$.  Let $c$ be a positive constant. \\   
Prove or disprove that  $f(n) + c\cdot g(k) = O(n+k)$. \\


{\color{blue}{\bf Answer:\\}}


(a) \\

(b) 






\newpage



\item  
Let $f(n) = \sum^n_{y{=}1} (n^6\cdot y^{23})$. 

Find a simple $g(n)$ such that $f(n) = \Theta(g(n))$, by proving that 
$f(n)=O(g(n))$, and that $f(n) = \Omega(g(n))$.

Don't use induction / substitution, or calculus, or any fancy formulas.  \\
Just exaggerate and simplify for big-O, then underestimate and simplify for $\Omega$.\\


{\color{blue}{\bf Answer:\\}}


\end{enumerate} 




\end{document}

